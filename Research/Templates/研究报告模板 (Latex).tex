% \documentclass[preprint,5p,twocolumn]{elsarticle}
\documentclass[preprint,3p]{elsarticle}
\usepackage{amssymb}
\usepackage{graphicx}
\usepackage{amsmath}
\usepackage{amsthm}
\usepackage{epstopdf}
\usepackage{epsfig}
\usepackage{listings}
\lstset{
  basicstyle=\ttfamily,
  columns=fullflexible,
  keepspaces=true,
  upquote=true,
  breaklines=true
}
\usepackage{xcolor}
\usepackage{float}
\biboptions{numbers,sort&compress}
\newtheorem{definition}{Definition}
\newtheorem{theorem}{Theorem}
\newtheorem{lemma}{Lemma}
\newtheorem{corollary}{Corollary}
\newtheorem{example}{Example}
\newtheorem{remark}{Remark}
\newtheorem{property}{Property}
\newtheorem{proposition}{Proposition}
\usepackage{xeCJK}
% https://github.com/wordshub/free-font
\setCJKmainfont[AutoFakeBold=true]{SourceHanSerifCN-Regular.ttf}
\newCJKfontfamily\Fangsong{Fangzheng-Fangsong.ttf}
\usepackage{setspace}

\begin{document}

\title{我的标题}
\makeatletter\let\Title\@title\makeatother

\begin{titlepage}
    \begin{center}
        \begin{figure}[H]
            \centering
            \includegraphics[width=2.33in]{QUN-logo.png}
        \end{figure}
        \textbf{\Huge 研~究~生~课~程~论~文} \\
        \vspace{0.1in} (2022—2023学年第2学期)
    \end{center}
    \vspace{0.2in} \textbf{\LARGE \raggedleft 题目:\Title}
    \begin{center}
        \vspace{0.2in} \textbf{\LARGE 说明} \\
    \end{center}

\begin{doublespace}
{\CJKfamily{Fangsong} \noindent
1. 课程论文要有题目、作者姓名、摘要、关键词、正文及参考文献。格式规范可参考《青海民族大学研究生论文编写规范》。 \\
2. 课程论文应为本人原创撰写,研究生院将随机抽查各学院课程论文并进行复制比检测,如发现论文是从网上下载或者抄袭剽窃他人文章的,按作弊处理,本门课程考核成绩计为0分,随下一年级重修。 \\
3. 课程论文应及时提交纸质版,任课教师进行评阅并打分后,及时提交学院归档保存。 \\
4. 论文题目、篇幅、内容等由任课教师做出具体要求。
}   
\end{doublespace}

\end{titlepage}


\newpage

\begin{frontmatter}  % article information
% title information
\title

% author information
\author{作者1}
\author{作者2}

% abstract & keywords
\begin{abstract}
摘要。
\end{abstract}

\begin{keyword}
关键词1 \sep 关键词2
\end{keyword}

\end{frontmatter}

\section{一级标题}

正文。


\bibliographystyle{plain}
\bibliography{ref.bib}

\newpage
\appendix

\end{document}